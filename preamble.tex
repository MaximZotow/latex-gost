\usepackage{blindtext} % использовалось для тестирований

% для использования с xe(la)tex
\usepackage{fontspec}
\usepackage[english,russian]{babel}
\setmainfont{Times New Roman} %\setmonofont{JetBrains Mono} % a font with cyrillic

% изменить название содержания
\addto\captionsrussian{\renewcommand{\contentsname}{СОДЕРЖАНИЕ}}

% 14 шрифт в документы
\usepackage[14pt]{extsizes}

% изменение стиля содержания
\usepackage{tocloft}
\renewcommand{\cftsecfont}{\normalfont}
\renewcommand{\cftsecleader}{\cftdotfill{\cftdotsep}}
\renewcommand{\cftsecpagefont}{\normalsize}
\setlength{\cftbeforesecskip}{0cm}
\setlength{\cftbeforesubsecskip}{0cm}
\setlength{\cftbeforesubsubsecskip}{0cm}
\setlength{\cftbeforeparaskip}{0cm}
\renewcommand{\cfttoctitlefont}{\hfill\large\bfseries}
\renewcommand{\cftaftertoctitle}{\hfill\hfill}

% добавляем параграфы в содержание и добавляем нумерацию
\setcounter{tocdepth}{5}
\setcounter{secnumdepth}{5}


% математика
\usepackage{amsmath,amssymb,amsfonts,icomma,mathtools}

% картинки (в квардратные скобки можно прописать draft и картинки будут добавляться ссылками, компиляция быстрее)
\usepackage[]{graphicx}

% гиперссылки и ссылки в документе (плюс формат ссылок Times New Roman'ом)
\usepackage[unicode]{hyperref}
\hypersetup{
	colorlinks,
	citecolor=black,
	filecolor=black,
	linkcolor=black,
	urlcolor=blue
}
\usepackage{url}
\urlstyle{same}

% размер шрифта разделов (нормальный)
\usepackage{sectsty}
\allsectionsfont{\bfseries\normalsize}

% изменяем стиль paragraph/subparagraph
\makeatletter
\renewcommand{\paragraph}{\@startsection{paragraph}{4}{0ex}%
	{-3.25ex plus -1ex minus -0.2ex}%
	{1.5ex plus 0.2ex}%
	{\normalfont\large\bfseries}}
\makeatother

\makeatletter
\renewcommand{\subparagraph}{\@startsection{subparagraph}{4}{0ex}%
	{-3.25ex plus -1ex minus -0.2ex}%
	{1.5ex plus 0.2ex}%
	{\normalfont\large\bfseries}}
\makeatother

% размер полей (по ГОСТу)
\usepackage[left=3cm,right=1cm, top=2cm, bottom=2cm]{geometry}

% полуторный интервал между строками
\usepackage{setspace}
\setstretch{1.5}

% отступ с первой строки
\usepackage{indentfirst}

% библиография по госту
\usepackage[
backend=biber,
maxbibnames=5,
natbib=true,
style=gost-numeric,
citestyle=numeric-comp,
sorting=none,
sortcites=true
]{biblatex}
%\bibliography{bibfile}


% команды, которые будут запущены в начале каждого документа
\AtBeginDocument{%
	\graphicspath{{../images},{images}} % первое для сабфайлов, второе для основного
}

% команды к конце каждого документа (полезно для subfiles)
\AtEndDocument{%
	\clearpage
	\newpage
	\sloppy
	\printbibliography[heading=bibintoc, title=СПИСОК ИСПОЛЬЗОВАННЫХ ИСТОЧНИКОВ]
	\label{verylastpage}
}

% для расширения функциональности таблиц
\usepackage{array}

% вставка листингов кода
\usepackage[final]{listings}
\usepackage{color} % создание собственной цветовой схемы
\definecolor{backcolour}{rgb}{0.95,0.95,0.92}
\definecolor{codegreen}{rgb}{0,0.6,0}

\lstdefinestyle{myStyle}{
	backgroundcolor=\color{backcolour},   
	commentstyle=\color{codegreen},
	basicstyle=\ttfamily\footnotesize,
	breakatwhitespace=false,         
	breaklines=true,                 
	keepspaces=true,                 
	numbers=left,       
	numbersep=5pt,                  
	showspaces=false,                
	showstringspaces=false,
	showtabs=false,                  
	tabsize=2,
}
\lstset{style=myStyle}

% автоподсчёт флотов в реферате
\usepackage{xspace}
\usepackage[figure,table,equation,lstlisting]{totalcount}

% считать количество элементов в библиографии
\usepackage{totcount}
\newtotcounter{citenum}
\AtEveryBibitem{\stepcounter{citenum}}

% свои команды
\newcommand{\ten}[1]{\cdot 10^{#1}} % возводить в степень
\newcolumntype{C}[1]{>{\centering\arraybackslash}m{#1}} % новый тип колонки в таблицах с центрированием по высоте и ширине
\newcommand{\rbf}[1]{\textbf{\ref{#1}}} % жирное цитирование
\newcommand{\refpar}[1]{\S\rbf{#1} на стр. \textbf{\pageref{#1}}} % цитирование номера параграфа с указанием на страницу
\newcommand{\refris}[1]{рис. \rbf{#1}} % ссылка на рисунок
\newcommand{\reftab}[1]{табл. \rbf{#1}} % ссылка на таблицу
\newcommand{\rt}[1]{_\text{#1}} % писать русский и не только текст в подстрочном режиме в математике


% субфигуры (много картинок в одной)
\usepackage[labelformat=simple, labelformat=brace]{subcaption}
\captionsetup{subrefformat=parens}
\renewcommand{\thesubfigure}{\asbuk{subfigure}}

% настройка подписей к картинкам, таблицам и прочее
\usepackage{floatrow}
\floatsetup[table]{capposition=top}
\usepackage[labelsep=period]{caption}
\captionsetup[table]{justification=centering,singlelinecheck=false,position=top}
\setlength{\abovedisplayskip}{3pt}
\setlength{\belowdisplayskip}{3pt}

% настройка списков (нумерованных и bullet) (НУЖЕН РЕФАКТОРИНГ)
\usepackage{enumitem}
\renewcommand{\labelitemi}{---}
\usepackage{enumitem}

\setlist{nolistsep} % убираем дополнительные вертикальные отступы вокруг списков

\setenumerate[1]{itemindent=0cm,
listparindent=\parindent} % для нумерованного списка
\setenumerate[2]{itemindent=0cm,
listparindent=2\parindent,leftmargin=2\parindent}

\renewcommand{\labelenumii}{\theenumii} % стиль первого слоя
\renewcommand{\theenumii}{\theenumi.\arabic{enumii}.} % стиль второго слоя

\setitemize[1]{itemindent=0cm,
listparindent=\parindent} % для bullet
\setitemize[2]{itemindent=0cm,
listparindent=2\parindent,leftmargin=2\parindent}

% использование множества файлов с возможностью отдельной компиляции
\usepackage{xr}
\usepackage{subfiles}
\addbibresource{\subfix{bibfile.bib}} % добавил файл с библиографией
\externaldocument{\subfix{template.tex}}
