\documentclass[../template.tex]{subfiles}
\begin{document}
	
\begin{center}
		\MakeUppercase{\bfseries\normalsize РЕФЕРАТ}\\
	на расчетно-пояснительную записку к курсовому проекту на тему:\\
	\bfseries<<Технология формирования трёхмерного фотонного кристалла>>
\end{center}	

Расчётно-пояснительная записка  содержит \pageref{verylastpage} страниц, \totaltables \xspace таблиц, \totalfigures \xspace рисунка, \totalequations \xspace уравнение, список литературы из \total{citenum} источников.


ФОТОННЫЙ КРИСТАЛЛ, СЕДИМЕНТАЦИЯ, ЭЛЕКТРОФОРЕЗ, ЦЕНТРИФУГИРОВАНИЕ, ИЗГОТОВЛЕНИЕ ФОТОННОГО КРИСТАЛЛА, МЕТОДО ШТОБЕРА, ТРЁХМЕРНЫЙ ФОТОННЫЙ КРИСТАЛЛ, МЕТОДЫ НАНЕСЕНИЯ ПРОВОДЯЩЕГО СЛОЯ, БЕЗПОРОГОВЫЙ ЛАЗЕР, ПОЛНЫЙ ФАКТОРНЫЙ ЭКСПЕРИМЕНТ

Работа посвящена разработке технологии изготовления фотонного кристалла с использованием методов осаждения.

В пояснительной записке к курсовой работе представлен технологический анализ трёхмерного коллоидного фотонного кристалла, а также анализ методов его изготовления.

Рассмотрен технологический маршрут изготовления трёхмерного фотонного кристалла. Проанализированы методы формирования опаловой структуры. Проведена металлизация фотонного кристалла. 

Построена математическая модель зависимости  ширины запрещённой зоны от мощности, подаваемой на магнетрон, и времени нанесения.

Предложен вариант оснастки, необходимой для проведения процесса нанесения металлизации на опаловую структуру.


Графические работы в объеме 5 листов выполнены на ПК с помощью программ Compas v21 и AutoDesk Inventor Professional 2020. Математическое моделирование выполнено инструментами GNU Octave и LibreOffice Calc. Записка выполнена с использованием среды \LaTeX.


	
	
\end{document}
